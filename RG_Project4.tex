\documentclass{article}
\usepackage[utf8]{inputenc}
\usepackage{amsmath}
\usepackage[]{algorithm2e}

\title{Random Graphs Project 4}
\author{Alex Rose, Pedro Rodriguez}
\date{April 2016}

\begin{document}

\maketitle

\section{ Graphs with Fixed Degree Distributions}

\subsection{ }

$$P(E_n) = pP(E_{n-1}) + (1-p)P(\neg E_{n-1}) = pP(E_{n-1}) + (1-p)(1- P(E_{n-1}) $$
$$ = (2p-1)P(E_{n-1}) + (1 - p), P(E_0) = p$$

\subsection{} 
Let $f(x) = (2p -1)x + (1-p)$. Note that for any real $x_1,x_0$:
$$|f(x_1) - f(x_0)| = |(2p-1)x_1 + (1-p) - (2p-1)x_0 - (1-p)| = |(2p-1)(x_1-x_0)|$$
This quantity is strictly less than $|x_1-x_0|$ if $0 < p < 1$, so f(x) is a contraction on that interval. So by the Contraction Mapping Theorem, the recursion will converge to f(x)'s unique fixed point on $0 < p < 1$ .  Solving for this fixed point:
$$x = (2p-1)x + (1-p)$$
$$2x - 2px = 1-p$$
$$x = \frac{1-p}{2(1-p)} = 1/2$$

So $\lim_{n\to\infty} P(E_{n})= x = 1/2$

\subsection{}
Uh, each step of the process removes 2 "degrees" from the degree list to form 1 edge. We have D degrees, so this process will form $D/2$ edges in total.

\subsection{}
With the first procedure, the probability of any degree being even is $p$ for large $n$ \footnote{handwave handwave handwave}. The second procedure generates an unrestricted $n-1$ sized degree list; each of these degrees will be even with exactly probability $p$, and their sum will be even with probability $1/2$ for large $n$. But the final restricted degree must match the eveness of the unrestricted degrees' sum, so it'll be even with probability $1/2$ \textit{regardless of p}. This restricted degree can a different distribution than the degrees in procedure (and will if $p \ne 1/2$), and so the two procedures are not equivalent.


\subsection{} 

\begin{algorithm}[H]
 initialize an NxN matrix Edges and an N length vector Degrees\;
 \While{Degrees is unset}{
 run simulate(d) for the n degree values in Degrees\;
  \If{sum of the degree values is even}{
	set Degrees to the generated degree values  \;
	\#(And exit the loop)
   }
 }
T = sum of all degrees in Degrees\;
\While{T \textgreater 0}{
	M, P = two nodes picked at random and weighted by their degree\;
	Form (M, P) by incrementing the corresponding values in Edges\;
	Decrement the Nth and Mth value in Degrees\;
	Decrement T by 2\;
 }
\Return Edges

\end{algorithm}

\subsection{}
Let $S_t = d_1 + d_2 + ... + d_t$. Then:

$$q_n  = \begin{cases}
  P(E_{n-t}), & \text{if } S \text{ is even}, \\
  1 - P(E_{n-t}), & \text{if }  S \text{ is odd}.
\end{cases}$$

As $t$ is finite, we have:
$$\lim_{n\to\infty} P(E_{n-t}) = \lim_{n\to\infty} P(E_{n})  = 1/2$$. 

So $\lim_{n\to\infty} q_n = 1/2$ in both cases.

\subsection{}
Let $D_t$ denote event $(d_1 = k_1, d_2 = k_2,..., d_t = k_t)$. By Baye's Theorem:

$$ \lim_{n\to\infty} P(D_t | E_n) = \lim_{n\to\infty} \frac{P(E_n | D_t)P(D_t)}{P(E_n)} = \frac{1/2* P(D_t)}{1/2} = P(D_t) $$

And the variables $d_1...d_n$ are all independent by assumption, so:
$$P(D_t) = P(d_1 = k_1) P(d_2 = k_2) ... P(d_t = k_t) = \prod_{j=1}^{t} P(d_j)$$ 

\end{document}
